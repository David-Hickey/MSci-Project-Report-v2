Atrial fibrillation is a heart rhythm disorder characterised by an elevated heart rate and ineffective contraction of the atria. In normal heart function, electrical signals are produced by the sinoatrial node and propagate through the myocardium with a single cohesive wavefront. Atrial fibrillation arises when electrical signals are produced in regions of the atrium outside the sinoatrial node. Typical treatments use surgical interventions to destroy the regions of tissue responsible for these anomalous electrical signals, but have a disappointing success rate due to poor understanding of the mechanisms of the disease. 

We developed a procedure to generalise an existing two-dimensional cellular automaton of atrial fibrillation to a more realistic three-dimensional substrate based on a sheep atrium. We demonstrated the feasibility of this generalisation, and showed that core results of the existing cellular automaton were preserved. 
We found that both the propagation of electrical signals and the spatial distribution of electrical anomalies in
our model were consistent with clinical observations.