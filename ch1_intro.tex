Atrial fibrillation (AF) is the most common heart rhythm disorder~\cite{narayan2012treatment}, and causes parts of the heart to spasm (or `fibrillate') instead of contracting effectively~\cite{nattel}. It affects predominantly the elderly~\cite{wasmer2017predisposing}. As of 2010 it affected 8.8 million people over the age of 50 in the EU, and this number is expected to double by 2050~\cite{krijthe2013projections}. AF causes a five-fold increase in stroke risk, is responsible for 25\% of strokes in people over the age of 80~\cite{wolf1991probability}, and the strokes caused by AF are more severe than strokes from other causes~\cite{miller2005cost}. The direct cost of AF to the NHS was \pounds 459~million in 2000~\cite{stewart2004cost}. 
Typically treatments take the form of surgical interventions, but as AF has poorly-understood mechanisms these surgical interventions have a high risk of complications and a low success rate~\cite{deshmukh2013hospital, cappato2005worldwide}.

The \cmp (CMP) model is a simple model of AF. It treats the heart as a two-dimensional discrete square lattice, where each site represents a heart cell. The state of each cell evolves over time, controlled by a set of simple rules which are based on the behaviour of real heart cells. These simple rules enable fibrillation to arise spontaneously~\cite{cmp}.
	
\section{Aims and Objectives}

In this work, we create a computational model of AF based on information about the cardiac muscle configuration (a `fibre map') of a real sheep heart, in which heart cells act based on the rules defined by the CMP model. Our specific aims are as follows:
\begin{enumerate}
    \item Demonstrate the feasibility of generalising the CMP model to a more realistic three-dimensional anatomy based on a sheep heart. We intend to move away from the lattice structure imposed by the CMP model, and towards a complex network structure possessing large-scale structural anisotropy reflective of a real sheep heart. We hope to verify that the core results of the original model are preserved in this generalisation.
    \item Determine the spatial distribution of electrical anomalies that drive AF and compare to the clinically measured distribution.
    \item Find computationally efficient ways to locate these electrical anomalies.
    \item Assess future clinical applications of this model, with a focus on whether this model can guide surgical interventions to improve their efficacy.
\end{enumerate}

\section{Related Works}

This project builds on or extends several related works:
\begin{itemize}
    \item \etalcite{cmp}{Christensen} defined the simple rules for cell behaviour that we adapt to a more complex geometry. 
    \item \etalcite{zhao}{Zhao} produced the fibre map from a sheep heart, which we are using to guide the creation of this geometry. They also designed a model of atrial fibrillation, but modelled the heart as a continuum in which electrical impulses are conducted with anisotropic propagation speeds. In comparison, our approach phenomenologically models individual cells.
    % \item The Master's thesis of Menkus~\cite{menkus} also built on this atrial fibre map but retained the discrete lattice structure of the CMP model; we discard this constraint, thereby solving several issues in the way electrical signals travel through the atrium.
\end{itemize}

\section{Report Outline}

The outline of the remainder of this report is as follows:
\begin{description}
    \item[Ch. \ref{ch:biologicalbackground}: \nameref{ch:biologicalbackground}] summarises the biology necessary to understand the rules behind the modelling process. It ends with a brief summary of some of the modelling approaches that have been used to study AF.
    \item[Ch. \ref{ch:cmpmodel}: \nameref{ch:cmpmodel}] describes the simple model of AF developed by \etalcite{cmp}{Christensen}. It also describes the extensions we have made to this simple model in preparation for our generalised model.
    \item[Ch. \ref{ch:generalisation}: \nameref{ch:generalisation}] details our work generalising the \cmp model to a more realistic substrate. Our procedure to create the substrate for our generalised model is explained and justified. The results shown by our model are shown and discussed, with a focus on whether they are consistent with AF in real hearts. This chapter represents the bulk of the novel work in this project.
    \item[Ch. \ref{ch:conclusion}: \nameref{ch:conclusion}] summarises our findings and discusses the clinical implications of our model. The limitations and future research potential for our model are considered.
\end{description}